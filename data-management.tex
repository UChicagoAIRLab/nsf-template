\begin{center}
{\large \bf \TITLE}
\end{center}

\begin{center}
    {\bf Data Management Plan}
\end{center}

Our proposed research, outreach, and educational activities may handle a
variety of datasets and data models.  The data used in the project will be
delineated between \emph{private data} that is unsuitable for any release and
must be access-controlled (e.g.,personally-identifying information, as
specified by an IRB, or private company data that is protected by NDA); and
\emph{public data} that is already broadly available.

\paragraph{Data Storage and Retention.} We will use the University of
Chicago secure data enclave (\url{https://securedata.uchicago.edu}), which
allows researchers to securely store and analyze sensitive research data
according to various restrictions, including local, state, federal, and
international laws. The facility provides best-in-class data security
standards for the storage of sensitive data, as well as the ability to
analyze sensitive data directly within the enclave, using software
researchers install and co-located compute facilities, including the
Research Computing Center's MidwayR computing cluster that is equipped
with tools and software that provide compliance with state-of-the-art
data protection.

Data will be guaranteed to be stored for the lifetime of the proposed
project, but in general we expect to store most data longer. Storage
beyond the end of the project period will be at the data owner's (see
below for definition) and data generator's discretion.

We will use any standard collaboration/communication tools to transmit
any data that can be released to those who request it. In general,
private data will not be transferable from one site to another; if the
need arises to transfer data between institutions, we will consult with
our corresponding university IRBs and use suitable data protection
mechanisms to transfer that data.

\paragraph{Data Ownership}. We find it helpful to ensure that datasets
always have a designated ``owner''. Although this person may or may
not have been the generator of the data, they will be responsible for
ensuring proper access control in the case of private data, facilitating
timely responses to requests for releasable data, and for ensuring
continued access for public
data. The owner will be the co-PI or senior researcher most closely
affiliated with the activity from which the data resulted. Should that
co-PI leave their institution or be, for some other reason, unable to
continue being owner of the data, they will either pass ownership to
another co-PI for the project or to another senior researcher at their
university.

In cases of handling sensitive data from one of our partners, decisions
regarding data management will be governed by corresponding negotiation
of data use agreements with our corresponding universities. For example,
the University of Chicago has a standard Data Use Agreement (DUA)
template that is used for incoming and outgoing data and managed by the
university research administration
(\url{https://ura.uchicago.edu/page/data-sharing-agreements})
to ensure that proper procedures and protocols are implemented for
transferring and protecting sensitive data. A critical first step will
be careful consultation with partners and other university officials
(e.g., university IRBs, university research administration and contracts
offices) regarding the data that will be shared, to inform policy about
the status of the data.

\paragraph{Primary Data.} This project will involve the collection and sharing
of various data concerning Internet connectivity, uptake, and student
engagement and achievement. When appropriate, and as described below, we aim
to make these datasets available through the Chicago Data
Portal~\cite{www-chicago-data}, in cooperation with our partners. In other
cases---in particular, for individual-level student data from the Chicago Public
Schools---certain sensitive data has been (or will be) shared under
restrictive data use agreements and non-disclosure agreements and cannot be
made public.  The datasets we plan to collect and/or analyze include the
following:
\begin{itemize}
    \itemsep=-1pt
    \item {\em Network performance data.} We will use existing tools and
        software, including the BISmark network measurement suite to gather
        continuous data about network performance, including throughput,
        latency, packet loss, and home wireless network performance.  All
        performance data will be made publicly available, except for any data
        pertaining to specific devices in a house that could be linked to an
        individual (e.g., IP addresses and hardware addresses will be
        appropriately anonymized).
    \item {\em Application quality data.} We will use the existing
        NetMicroscope toolkit that we have developed to measure and infer
        application quality and user experience for a broad array of
        applications that Chicago Public School students depend on for online
        learning. As with network performance data, and identifying
        information, such as IP addresses and hardware addresses, will be
        appropriately anonymized before public release.
    \item {\em Application and Internet usage data.} Our ability to observe
        traffic inside of the home makes it possible to gain insight into
        Internet usage in the home, including what applications and services
        were being used at what times. This data is of a more sensitive nature
        and we do not plan to publicly release it in raw form. Rather, we will
        release data including aggregate statistics that are derived from the
        raw data, such as the breakdown of application and device usage across
        all homes, the typical numbers of simultaneous Internet users in a
        household, and so forth.
\end{itemize}

\paragraph{Other Data.} Several other kinds of data will be generated in the
course of this project:

\begin{itemize}
\item
  \emph{Source code}: We will make all source code open
  source under a suitable nonrestrictive license.
\item
  \emph{Educational materials}: As discussed in the proposal, we plan to
        create course modules and integrate these materials into both new an
        existing courses. Whenever possible, we will release course
        materials---lecture and assignment materials, videos, etc.---for other
        educators to integrate into their own courses.
\end{itemize}
