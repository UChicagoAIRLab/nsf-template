\section{Broader Impacts}\label{sec:impacts}

The project has two broader impact goals: (1) improving DNS privacy through
the deployment of new protocols in practice; (2) educating a diverse set of
next-generation computer scientists. In this section, we highlight 
initiatives toward each of these goals.

\paragraph{Improving DNS privacy in practice.}
We aim to transition research results to practice, through
continued engagement in standards bodies (\eg, Internet Engineering Task
Force) and operators groups (\eg, North American Network Operators Group) and
through 
integration with browsers (\eg, Firefox, Tor
Browser) and open-source stub resolvers (\eg, Stubby). We have created an
initial prototype implementation of a distributed DNS stub resolver as an
extension to Facebook's open-source DoH proxy~\cite{fb-doh-proxy}. In this
project, we will aim for more comprehensive and widespread open-source
deployments, building on early discussions with both Mozilla and the Tor
Project. The PI plans to continue to engage with the IETF DNS Privacy
(DPRIVE) working group~\cite{dprive} to continue to develop and propose standards based on
research outcomes.

\paragraph{Student and researcher mentorship.} To improve education in ways
that broaden participation in computing, the PI will create interconnected
programs that merge sustained engagement with diverse students in Chicago
Public Schools, provide summer research experiences, and carefully scaffold
these research experiences with on-ramp immersions that lower barriers to
participation.  

To provide meaningful research experiences to students, PI Feamster will
annually mentor a diverse group of students, ranging in level from high school
to post-graduate as a way to encourage future participation in computing
programs. These efforts will benefit from engaged partnerships with
organizations to aid in recruitment and tracking student outcomes: (1)~The
UChicago Collegiate Scholars Program; (2) the Office of Special Programs (OSP)
College Prep; (3)~The Fisk-Vanderbilt Master's-to-PhD Bridge Program for
underrepresented students.

The PI has a successful track record in mentoring women and underrepresented
minorities. For example, he currently advises two female Ph.D. students, one
of whom is also an underrepresented minority. These students have won research
awards; Annie Edmundson won the 2016 ACM Student Research Competition at ACM
SIGCOMM. The PI is involved with various efforts around the department to
increase inclusion and diversity.  Feamster also regularly involves
undergraduate students in research, both during the academic year and summers.
PI Feamster mentored three undergraduate students (including one female
student) in Spring 2017 on a home network IoT inspector project, which
resulted in a submission to the Federal Trade Commission's IoT Home Network
Challenge~\cite{iot-monitor-ftc}, a presentation to the Coalition for National
Science Funding~(CNSF)~\cite{cnsf2017-iot}, and presentations to Congress,
including Senator Markey~(MA) and Representative McNerney~(CA-09).  In
Summer~2017, Feamster hosted several undergraduates for research, including
Jessica May, a female undergraduate student from the University of Nebraska.
The research performed by undergraduates has often resulted in peer-reviewed
publications (including as first author on top-tier venues such as SIGCOMM and
IMC).  Many of the PI's undergraduates have continued either to top graduate
programs or highly desirable companies. Of those, in the past few years, PI
Feamster has worked with six female research assistants.  Another aspect of
education is providing students with internship opportunities The PI will seek
out real-world opportunities at government offices, companies, and civil
society organizations to train students.  When hiring, the PI will follow best
practices for encouraging diversity in hiring, including interviewing a
diverse set of candidates and advertising in forums that serve
under-represented minorities (\eg, Tapia and Grace Hopper).

\paragraph{Outreach to under-represented high school students.}
To further engage under-represented minorities in the project, the PIs will
integrate results from this project into K-12 education programs, such as the
University of Chicago's Upward Bound program, where PI Feamster currently
teaches an ``Applications of Machine Learning to Networking'' summer course to
under-represented students across the Chicago Public School system.
This program  has allowed the PI to directly engage with high
school students  enrolled in Chicago Public Schools (77\% economically
disadvantaged and 83\% underrepresented minority) and their parents for an
annual series of 10 seminars and 6 weekend workshops hosted at UChicago on our
project's key themes. 
This on-ramp will introduce core computing
skills, build an understanding of the research process (e.g., how to ask a
good research question), and help the students set productive routines in
unstructured environments.  Perhaps more importantly, it will help build a
community and support network among each year's student cohort and the larger
team of co-PIs through near-peer mentoring and informal interactions.
Following this immersion, the students will work for the rest of the summer
with their respective co-PIs.

\paragraph{Course development.} PI Feamster is currently teaching a research
seminar on the Internet and free expression and drafting a book on the topic.
The PI will develop new course materials pertaining to DNS privacy, both for
courses at the University of Chicago and in the form of publicly accessible
course material, as in Massive Open Online Courses (MOOCs), which PI Feamster
has developed in the past. The PI already teaches several courses where topics
from the research may be integrated, including undergraduate Information
Security, Networking, Many University of Chicago undergraduate students
also perform independent research. The PI will generate project ideas for
independent work projects and undergraduate theses and will invite students to
undertake these projects.


\paragraph{Integration into educational programs.}
Ultimately, we aim to integrate the data that we gather into educational
materials and courses. Examples include Machine Learning for Public Policy;
Data Science, Social Science and Public Policy; and the Machine Learning
Concepts and Applications course that PI Feamster teaches in the University of
Chicago's Upward Bound, a college readiness program for under-represented
minorities in the Chicago Public School system. The Center for Data and
Computing will also involve industry partners in specific projects and
engagements with students through the Chicago Data Science and Technology
Clinic, a new program that matches computer science and computer science
masters students to longer-term projects, such as those we describe in this
proposal.
The PI will also convene student
activities, including hackathons, that leverage the outputs from the proposed
research and are geared towards outputs with broader impact.  We plan to help
organize local student activities that help students prepare
entries for these competitions at the Center for Data and Computing (CDAC), as we have in the past. 


